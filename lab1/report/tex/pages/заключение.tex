\section*{Заключение}
\addcontentsline{toc}{section}{Заключение}

В ходе выполнения лабораторной работы была разработана программа для работы с мультимножествами на основе кода Грея. Реализованы все требуемые функции: генерация кода Грея произвольной разрядности, создание универсума и мультимножеств, операции над мультимножествами (объединение, пересечение, разность, симметрическая разность, дополнение), арифметические операции над кратностями (сумма, разность, произведение, деление), а также сравнение мультимножеств на равенство.

\subsection*{Достоинства реализации}

\begin{itemize}
    \item Объектно-ориентированная архитектура с чётким разделением ответственности между классами \texttt{Universe}, \texttt{Multiset} и \texttt{CLIUI}.
    \item Использование современных возможностей C++17: умные указатели (\texttt{std::unique\_ptr}), библиотека \texttt{<chrono>} для измерения времени.
    \item Адаптивный вывод: компактные таблицы для небольших множеств, постраничный режим для больших.
    \item Возможность сохранения результатов операций как новых мультимножеств для дальнейшего использования.
    \item Измерение и отображение времени выполнения операций с предупреждениями о долгих вычислениях.
\end{itemize}

\subsection*{Недостатки реализации}

\begin{itemize}
    \item Линейный поиск в методе \texttt{contains} класса \texttt{Universe} имеет сложность $O(n)$, что может быть оптимизировано использованием хеш-таблицы.
    \item Созданные мультимножества не сохраняются между сеансами работы программы.
    \item Все операции над мультимножествами (объединение, пересечение, разность, симметрическая разность, дополнение, арифметические операции) возвращают новые объекты, создавая копии данных. При работе с большими универсумами это увеличивает потребление памяти.
\end{itemize}

\subsection*{Масштабируемость}

Архитектура программы обеспечивает гибкость для дальнейшего развития.

\begin{itemize}
    \item Добавление новых операций над мультимножествами сводится к реализации методов в классе \texttt{Multiset} и обновлению меню в \texttt{CLIUI}. Базовая логика и структура данных остаются неизменными.
    
    \item Логика работы с мультимножествами полностью отделена от пользовательского интерфейса. Это позволяет заменить консольный интерфейс на графический или веб-интерфейс без изменения классов \texttt{Universe} и \texttt{Multiset}.
    
    \item Классы \texttt{Universe} и \texttt{Multiset} могут использоваться как библиотека в других проектах, требующих работы с мультимножествами и кодом Грея.
\end{itemize}

Таким образом, разработанная программа имеет возможности для масштабирования, что делает её пригодной для дальнейшего развития и использования в более сложных задачах, связанных с мультимножествами и кодом Грея.