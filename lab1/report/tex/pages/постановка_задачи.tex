\section{Постановка задачи}

В рамках лабораторной работы необходимо разработать программное обеспечение, которое продемонстрирует на практике работу с бинарным кодом Грея и мультимножествами. Программа должна быть интерактивной, устойчивой к ошибкам пользователя и обеспечивать понятный интерфейс для формирования и обработки множеств.

Цель данной работы~--- реализовать программу, которая:
\begin{itemize}
    \item генерирует бинарный код Грея заданной разрядности;
    \item формирует мультимножества двумя способами~--- ручным и автоматическим;
    \item выполняет над ними стандартные операции: объединение, пересечение, разность, симметрическую разность, дополнение;
    \item выполняет арифметические операции над кратностями: сумму, разность, произведение, деление;
    \item защищает пользователя от некорректного ввода;
    \item позволяет создавать несколько мультимножеств и выбирать из них те, над которыми будут проводиться операции.
\end{itemize}