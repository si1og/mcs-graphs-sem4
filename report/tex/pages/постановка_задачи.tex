\section{Постановка задачи}

В рамках курсовой работы необходимо реализовать калькулятор <<большой>> конечной арифметики $\langle Z_8; +, * \rangle$ (8 разрядов) для четырёх действий ($+$, $-$, $*$, $/$) на основе <<малой>> конечной арифметики, где задано правило <<$+1$>> и выполняются свойства коммутативности ($+$, $*$), ассоциативности ($+$, $*$), дистрибутивности $*$ относительно $+$, заданы аддитивная единица <<$a$>> и мультипликативная единица <<$b$>>, а также выполняется свойство: $(\forall x) \; x * a = a$.

Алфавит системы: $Z_8 = \{a, b, c, d, e, f, g, h\}$.

Правило <<$+1$>> для варианта 49 задаётся следующим образом:

\begin{table}[h]
\centering
\begin{tabular}{|c|c|c|c|c|c|c|c|c|}
\hline
$x$ & $a$ & $b$ & $c$ & $d$ & $e$ & $f$ & $g$ & $h$ \\
\hline
$x+1$ & $b$ & $g$ & $a$ & $h$ & $f$ & $c$ & $d$ & $e$ \\
\hline
\end{tabular}
\caption{Правило <<+1>> для варианта 49}
\end{table}

Данное правило определяет порядок элементов:
$$a \rightarrow b \rightarrow g \rightarrow d \rightarrow h \rightarrow e \rightarrow f \rightarrow c$$

Необходимо реализовать программу, которая:

\begin{itemize}
    \item поддерживает все четыре действия, а именно:
    \begin{itemize}
        \item сложение,
        \item вычитание,
        \item умножение,
        \item деление, в т.ч. с остатком;
    \end{itemize}
    
    \item поддерживает числа до 8 разрядов включительно (например, возможно выражение $bbb + ccc$);
    
    \item поддерживает отрицательные числа;
    
    \item выполняет вычисления только в рамках данной арифметики без перевода в иные системы счисления.
\end{itemize}