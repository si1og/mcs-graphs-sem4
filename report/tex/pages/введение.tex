\section*{Введение}
\addcontentsline{toc}{section}{Введение}

Конечная арифметика~--- раздел дискретной математики, изучающий алгебраические структуры с конечным числом элементов. В отличие от классической арифметики, где множество чисел бесконечно, здесь операции определены на ограниченном алфавите символов. Это свойство находит применение в криптографии, теории кодирования и проектировании цифровых устройств.

Малая конечная арифметика задаётся правилом <<$+1$>>, определяющим циклический порядок элементов алфавита. На её основе строится большая конечная арифметика, позволяющая работать с многоразрядными числами.

Цель данной работы~--- разработка калькулятора большой конечной арифметики, на основе малой конечной арифметики с заданным алфавитом символов и таблицей <<+1>>.